\begin{abstractslongpt}

% >1200 words


Desde os anos 60 que se tem observado uma explosão de informação biomédica proveniente sobretudo dos esforços de sequenciação dos genomas de diversos organismos. 
Neste contexto promoveu-se o desenvolvimento e o aperfeiçoamento de técnicas que permitiram o avanço e o estabelecimento de disciplinas da pós-genómica, como a proteómica e metabolómica, o que gerou ainda mais informação.
Assim surgiu um grande interesse pela Bioinformática como a disciplina que procura armazenar, gerir a analisar todos estes dados de interesse biológico.



\end{abstractslongpt}
